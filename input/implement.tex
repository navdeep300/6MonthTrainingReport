\section{Introduction to PHP}
\begin{figure}[h]
\centering \includegraphics[scale=0.4]{images/php.png}
\caption{Php logo}
\end{figure}
\noindent PHP is an open source server-side scripting language designed for Web development to produce dynamic Web pages. It is one of the first developed server-side scripting languages to be embedded into an HTML source document rather than calling an external file to process data. The code is interpreted by a Web server with a PHP processor module which generates the resulting Web page. It also has evolved to include a command-line interface capability and can be used in standalone graphical applications.\\

\noindent PHP can be deployed on most Web servers and also as a standalone shell on almost every operating system and platform, free of charge. A competitor to Microsoft’s Active Server Pages (ASP) server-side script engine and similar languages, PHP is installed on more than 20 million Web sites and 1 million Web servers. Notable software that uses PHP includes Drupal, Joomla, MediaWiki, and WordPress. PHP is a general-purpose scripting language.\\

\noindent It is especially suited to server-side web development where PHP generally runs on a web server. Any PHP code in a requested file is executed by the PHP runtime, usually to create dynamic web page content or dynamic images used on Web sites or elsewhere. It can also be used for command-line scripting and client-side graphical user interface (GUI) applications. PHP can be deployed on most Web servers, many operating systems.
\subsection{Features of PHP}
\begin{itemize}
\item Http Authentication
\item Cookies and Sessions
\item Connection Handling
\item Designer-friendly 
\item Cross platform Compatibility 
\item Loosely typed Language
\item Open Source
\item Easy code
\end{itemize}



\section{MySQL Database Server}
\begin{figure}[h]
\centering \includegraphics[scale=0.2]{images/mysql.jpg}
\caption{Mysql logo}
\end{figure}
\noindent I used the Mysql database for my project. It is world''s most popular open source database It 
is a relational database management system (RDBMS) that runs as a server 
providing multi-user access to a number of databases. It is named after 
developer Michael Widenius's daughter, My. The SQL phrase stands for
Structured Query Language. MySQL is written in C and C++.\\

\noindent Free-software-open source projects that require a 
full-featured database management system
often use MySQL. MySQL is also used in many high-profile, large-scale World 
Wide Web products, including
Wikipedia, Google (though not for searches) and Facebook.\\

\noindent MySQL is a popular choice of database for use in web 
applications, and is a central component of the widely used LAMP web 
application software LAMP is an acronym for “Linux, Apache, MySQL, 
Perl/PHP/Python”. MySQL is used in some of the most frequently visited web sites 
on the Internet, including Flickr, Nokia.com, YouTube, Wikipedia, Google 
and Facebook.\\

\noindent One of the greatest advantage of Django is that it synchronises the 
database only with one command withouut having any need to send 
different queries for insertion, deletion, updation etc. There is a 
file named models.py which is used for purpose of creating database.
\subsection{Features of MySQL}
\begin{itemize}
\item MySQL is a database management system.
\item MySQL is a relational database management system.
\item MySQL software is Open Source.
\item The MySQL Database Server is very fast, reliable, and easy to 
use.
\item MySQL Server works in client/server or embedded systems.
\item A large amount of contributed MySQL software is available.
\end{itemize}
\subsection{Installation of MySQL}
MySql can be installed using following commands:\\

\hspace{4pt} \$ sudo apt-get install mysql-server\\

\hspace{4pt} \$ sudo apt-get install mysql-client


\section{Introduction to Bootstrap} 

\begin{figure}[h]
\centering \includegraphics[scale=0.3]{images/bootstrap.png}
\caption{Bootstrap logo}
\end{figure}
\subsection{What is Bootstrap}
\noindent Bootstrap is a powerful front-end framework for faster and easier web development. It includes HTML and CSS based design templates for common user interface components like Typography, Forms, Buttons, Tables, Navigations, Dropdowns, Alerts, Modals, Tabs, Accordion, Carousel and many other as well as optional JavaScript extensions.
Bootstrap also gives you ability to create responsive layout with much less efforts.
\subsection{Advantages of Bootstrap}
The biggest advantage of using Bootstrap is that it comes with free set of tools for creating flexible and responsive web layouts as well as common interface components.
Additionally, using the Bootstrap data APIs you can create advanced interface components like Scrollspy and Typeaheads without writing a single line of JavaScript.
Here are some more advantages, why one should opt for Bootstrap:
\begin{itemize}
\item Save lots of time.
\item Responsive features.
\item Consistent design .
\item Easy to use.
\item Compatible with browsers.
\item Open Source.
\item Consistency.
\item Comprehensive List Of Components
\item Leveraging Javascript Libraries.
\item Frequent Updates.
\end{itemize}
\subsection{Installation of Bootstrap}
Downloading of Bootstrap is a very easy proccess.
Type the commands in the terminal:\\

 \$ git clone https://github.com/twbs/bootstrap.git\\


\noindent This will clone the bootstrap files on your pc/laptop and later u can use these files in your project.


\section{Introduction to Apache Web Server}

\begin{figure}[h]
\centering\includegraphics[scale=0.5]{images/apache.jpg}
\caption{Apache logo}
\end{figure}
\noindent Apache is a web server software notable for playing a key role in the initial 
growth of the World Wide Web. Apache is developed and maintained by an 
open community of developers under the auspices of the Apache Software 
Foundation. The application is available for a wide variety of operating 
systems, including Unix, FreeBSD, Linux, Solaris, Novell NetWare, Mac OS X, 
Microsoft Windows, OS/2, TPF, and eComStation. Released under the Apache 
License, Apache is open-source software.

\noindent The goal of this project is to provide a secure, efficient and extensible 
server that provides HTTP services in sync with the current HTTP standards.
\subsection{Features of Apache Server}
\begin{itemize}
\item Apache supports a variety of features, many implemented as compiled 
modules which extend the core functionality. These can range from 
server-side programming language support to authentication schemes. 
\item Apache features configurable error messages, DBMS-based 
authentication databases, and content negotiation. It is also supported 
by several graphical user interfaces (GUIs).
\item It supports password authentication and digital certificate 
authentication. Apache has a built in search engine and an HTML authorizing 
tool and supports FTP.
\end{itemize}

\subsection{Installation of Apache Server}
Apache web server can be installed using following commands:\\

\hspace{4pt} \$ sudo apt-get install apache2


\section{Introduction To Github}
\begin{figure}[h]
\centering \includegraphics[scale=0.27]{images/github.jpg}
\caption{Github logo}
\end{figure}
\noindent GitHub is a Git repository web-based hosting service which offers all of the functionality of Git as well as adding many of its own features. Unlike Git which is strictly a command-line tool, Github provides a web-based graphical interface and desktop as well as mobile integration. It also provides access control and several collaboration features such as wikis, task management, and bug tracking and feature requests for every project.\\

\noindent GitHub offers both paid plans for private repto handle everything from small to very large projects with speed and efficiency. ositories, and free accounts, which are usually used to host open source software projects. As of 2014, Github reports having over 3.4 million users, making it the largest code host in the world.\\

\noindent GitHub has become such a staple amongst the open-source development community that many developers have begun considering it a replacement for a conventional resume and some employers require applications to provide a link to and have an active contributing GitHub account in order to qualify for a job.\\\\

\section{What is Git?}
\begin{figure}[h]
\centering \includegraphics[scale=0.3]{images/git.jpg}
\caption{Git logo}
\end{figure}
\noindent Git is a distributed revision control and source code management (SCM) system with an emphasis on speed, data integrity, and support for distributed, non-linear workflows. Git was initially designed and developed by Linus Torvalds for Linux kernel development in 2005, and has since become the most widely adopted version control system for software development.\\

\noindent As with most other distributed revision control systems, and unlike most client–server systems, every Git working directory is a full-fledged repository with complete history and full version-tracking capabilities, independent of network access or a central server. Like the Linux kernel, Git is free and open source software distributed under the terms of the GNU General Public License version 2 to handle everything from small to very large projects with speed and efficiency.\\

\noindent Git is easy to learn and has a tiny footprint with lightning fast performance. It outclasses SCM tools like Subversion, CVS, Perforce, and ClearCase with features like cheap local branching, convenient staging areas, and multiple workflows.\\

\subsection{Installation of Git}

Installation of git is a very easy process.
The current git version is: 2.0.4.
Type the commands in the terminal:\\\\
\emph{
\$ sudo apt-get update\\\\
\$ sudo apt-get install git\\\\}
This will install the git on your pc or laptop.

\subsection{Various Git Commands}

Git is the open source distributed version control system that facilitates GitHub activities on your laptop or desktop. The commonly used Git command line instructions are:-\\

\subsection*{Create Repositories}
\addcontentsline{toc}{subsection}{Create Repositories}
Start a new repository or obtain from an exiting URL

\begin{description}

\item [\$ git init [ project-name]]\\
Creates a new local repository with the specified name
\item [\$ git clone [url]]\\
Downloads a project and its entire version history

\end{description}

\subsection*{Make Changes}
\addcontentsline{toc}{subsection}{Make Changes}
Review edits and craft a commit transaction

\begin{description}

\item [\$ git status] \leavevmode \\
Lists all new or modified files to be committed

\item [\$ git diff] \leavevmode \\
Shows file differences not yet staged

\item [\$ git add [file]]\\
Snapshots the file in preparation for versioning

\item [\$ git reset [file]]\\
Unstages the file, but preserve its contents

\item [\$ git commit -m "[descriptive message]"]\\
Records file snapshots permanently in version history

\end{description}

\subsection*{Group Changes}
\addcontentsline{toc}{subsection}{Group Changes}
Name a series of commits and combine completed efforts

\begin{description}

\item [\$ git branch] \leavevmode \\
Lists all local branches in the current repository

\item [\$ git branch [branch-name]]\\
Creates a new branch

\item [\$ git checkout [branch-name]]\\
Switches to the specified branch and updates the working directory

\item [\$ git merge [branch]]\\
Combines the specified branch’s history into the current branch

\item [\$ git branch -d [branch-name]]\\
Deletes the specified branch

\end{description}




%\section{Debconf}
%\input {input/debconf.tex}

\section{Introduction to \LaTeX}
\section{Introduction to \LaTeX}

\LaTeX, I have used this tool for preparing my six weeks training report and i found it excellent. So this time again i decided to use it for my report.
\LaTeX{} (pronounced /ˈleɪtɛk/, /ˈleɪtɛx/, /ˈlɑːtɛx/, or /ˈlɑːtɛk/) is a 
document markup language and document preparation system for the \TeX{} 
typesetting  program. Within the typesetting system, its name is styled 
as \LaTeX.

\image{0.9}{images/donald.png}{Donald Knuth, Inventor Of \TeX{} 
typesetting system}

\noindent Within the typesetting system, its name is styled as \LaTeX. The term 
\LaTeX{} refers only to the language in which documents are written, 
not to the editor used to write those documents. In order to create a 
document in \LaTeX, a .tex file must be created using some form of text 
editor. While most text editors can be used to create a \LaTeX{} document, 
a number of editors have been created specifically for working with \LaTeX.

\noindent \LaTeX{} is most widely used by mathematicians, scientists, 
engineers, philosophers, linguists, economists and other scholars in 
academia. As a primary or intermediate format, e.g., translating DocBook 
and other XML-based formats to PDF, \LaTeX{} is used because of the 
high quality of typesetting achievable by \TeX. The typesetting system 
offers programmable desktop publishing features and extensive facilities 
for automating most aspects of typesetting and desktop publishing, 
including numbering and cross-referencing, tables and figures, 
page layout and bibliographies.

\noindent \LaTeX{} is intended to provide a high-level language that
accesses the power of \TeX. \LaTeX{} essentially comprises a
collection of \TeX{} macros and a program to process \LaTeX documents. 
Because the \TeX{} formatting commands are very low-level, it is usually 
much simpler for end-users to use \LaTeX{}.


\section{Typesetting}
\LaTeX{} is based on the idea that authors should be able to focus on 
the content of what they are writing without being distracted by its 
visual presentation. In preparing a \LaTeX{} document, the author 
specifies the logical structure using familiar concepts such as 
chapter, section, table, figure, etc., and lets the \LaTeX{} system 
worry about the presentation of these structures. It therefore 
encourages the separation of layout from content while still allowing 
manual typesetting adjustments where needed. 

\begin{verbatim}
\documentclass[12pt]{article}
\usepackage{amsmath}
\title{\LaTeX}
\date{}
\begin{document}
  \maketitle 
  \LaTeX{} is a document preparation system 
  for the \TeX{} typesetting program.
   \par 
   $E=mc^2$
\end{document}
\end{verbatim}



\section{Introduction to Doxygen}
\begin{figure}[h]
\centering \includegraphics[scale=1]{images/doxygen.jpg}
\end{figure}
\noindent Doxygen is a documentation generator, a tool for writing software reference 
documentation. The documentation is written within code, and is thus 
relatively easy to keep up to date. Doxygen can cross reference 
documentation and code, so that the reader of a document can easily 
refer to the actual code.

\noindent Doxygen supports multiple programming languages, especially C++, C, 
C\#, Objective-C, Java, Python, IDL, VHDL, Fortran and PHP.[2] Doxygen
 is free software, released under the terms of the GNU General Public 
License.\\

\subsection{Features of Doxygen}
\begin{itemize}
\item Requires very little overhead from the writer of the documentation. 
Plain text will do, Markdown is support, and for more fancy or structured 
output HTML tags and/or some of doxygen's special commands can be used.
\item Cross platform: Works on Windows and many Unix flavors (including 
Linux and Mac OS X).
\item Comes with a GUI frontend (Doxywizard) to ease editing the options 
and run doxygen. The GUI is available on Windows, Linux, and Mac OS X.
\item Automatically generates class and collaboration diagrams in HTML 
(as clickable image maps) and $\mbox{\LaTeX}$ (as Encapsulated PostScript 
images).
\item Allows grouping of entities in modules and creating a hierarchy 
of modules.
\item Doxygen can generate a layout which you can use and edit to change 
the layout of each page.
\item Can cope with large projects easily.
\end{itemize}
\subsection{Installation of Doxygen}
Doxygen can be installed using following commands:\\

\hspace{4pt} \$ git clone https://github.com/doxygen/doxygen.git\\ 

\hspace{4pt} \$ cd doxygen\\

\hspace{4pt} \$ ./configure\\

\hspace{4pt} \$ make \\
\newpage



\newpage
\section{Implementation}
Implementation is the process of converting a new or revised system 
design into an operational one. At the present time there is no system 
as Imperial Finance which work online and provide information via web.
So this is the replacement of the manual financial system. In Imperial 
Finance most of the finance related task will be performed online.\\

{\bf Types of Implementation:}
\begin{enumerate}
\item Implementation of a computer system to replace a manual system.
\item Implementation of a new computer system to replace an existing one.
\item Implementation of a modified application to replace an existing one.\\
\end{enumerate} 

\hspace{0.0cm} {\bf Aspects of Implementation:}
\begin{enumerate}
\item Conversion
\item Post Implementation and review
\item Software maintenance
\end{enumerate}
\vskip 0.5cm
\subsection{Implementation of the Project }
Elearning System is the implementation of the new system to replace manual one. Working manually is very time consuming and irritating. The project 
implementation of starts with the Administrator. Administrator will be the super user of the application who will configure system information. There will be a different interface for the Students and Teachers from where they can manage and view the required information.\\

\noindent It is a web based application, so it is distributed and data centric. 
In this application, MySQL database is used to store data related to 
Students and Teachers. Since database is on 
Server, so any number of users can work simultaneously and can share 
their data with each other.
\subsection{Conversion Plan}
Conversion is the process of changing from one system to another. This 
plan involves:
\begin{enumerate}
\item Creating computer-compatible files.
\item Training the operating staff.
\item Installing terminals and hardware.
\end{enumerate}

\subsection{Conversion Processes}
\begin{enumerate}
\item File Conversion.
\item  Data Entry.
\item User Training.
\end{enumerate}
\vskip 0.5cm
\subsection{Elements of User training}
\begin{enumerate}
\item The initial training period.
\item At the time of Installation.
\item If required, during Maintenance Phase.
\end{enumerate}

\section{Post-Implementation and Software Maintenance}
Implementation review is an evaluation of a system in terms of the 
extent to which the system accomplishes stated objectives and actual 
project costs exceeds initial estimates.
\subsection{Review Plan}
An overall plan covers following aspects:
\begin{enumerate}
\item Administrative plan.
\item Personnel requirements plan.
\item Hardware plan.
\item Documentation review plan.
\end{enumerate}
\vskip 0.5cm
After the implementation of this project, the team will see the post 
implementation phase. If there will be any concerns, those will be 
solved based on the user feedback.
\subsection{Maintenance}
In order for a software system to remain useful in its environment it 
may be necessary to carry out a wide range of maintenance activities 
upon it. There are bugs to fix, enhancement to add and optimization to 
make, changes has to be done in older version to make it application 
for current use of current version to cater the need of future. 
Maintenance can be of three types:\\
\begin{enumerate}
\item {\bf{Corrective Maintenance}}: Changes necessitated by actual errors 
(defects or residual "bugs") in a system are termed corrective 
maintenance. These defects manifest themselves when the system does not
operate as it was designed or advertised to do. A defect or “bug” can 
result from design errors, logic errors and coding errors. Design errors 
occur when for example changes made to the software are incorrect, 
incomplete, wrongly communicated or the change request misunderstood. 
In the event of a system failure due to an error, actions are taken to 
restore operation of the software system. The approach here is to locate 
the original specifications in order to determine what the system was 
originally designed to do.
\item {\bf{Adaptive Maintenance}}: Any effort that is initiated as a result of 
changes in the environment in which a software system must operate is 
termed adaptive change. Adaptive change is a change driven by the need
to accommodate modifications in the environment of the software system, 
without which the system would become increasingly less useful until it 
became obsolete. The term environment in this context refers to all the 
conditions and influences which act from outside upon the system, for 
example business rules, government policies, work patterns, software 
and hardware operating platforms. A change to the whole or part of this 
environment will warrant a corresponding modification of the software.
\item {\bf{Perfective Maintenance}}: This is actually the most common type of 
maintenance encompassing enhancements both to the function and the 
efficiency of the code and includes all changes, insertions, deletions, 
modifications, extensions, and enhancements made to a system to meet 
the evolving and/or expanding needs of the user. A successful piece of 
software tends to be subjected to a succession of changes resulting in 
an increase in its requirements. This is based on the premise that as 
the software becomes useful, the users tend to experiment with new 
cases beyond the scope for which it was initially developed. Expansion 
in requirements can take the form of enhancement of existing system 
functionality or improvement in computational efficiency. Though 
efforts have been made to develop error free systems, but no system is 
perfect, room for improvement is always there. Thus proper documentation 
for the system has been done so that it will be easy to handle any 
breakdown or any other type of system maintenance activity.

\end{enumerate}

\section{Testing}
Project testing is an investigation conducted to determine the quality of the project and the services provided by the project. Testing is the process of analyzing a project to detect the differences between existing and required conditions (that is defects/errors/bugs) and to evaluate the features of the project. After complete development of the project it is mandatory to test the project. The main motive of the project testing is to identify whether project is able to meet user
requirements or not. To know the better performance of project we have to develop various Test Cases. Now, designing good test cases is a complex art. The complexity comes from three sources
\begin{itemize}
\item Test cases help us discover information. Different types of tests are more effective for different classes of information.
\item Test cases can be good in a variety of ways. No test case will be good in all of them.
\item Our tend to create test cases according to certain testing styles, such as domain testing or risk-based testing. Good domain tests are different from good risk-based tests.
\end{itemize}
\subsection{Unit Testing}
Unit testing is undertaken after a module has been coded and successfully reviewed. Unit testing (or module testing) is the testing of different units (or modules) of a system in isolation. I have done unit testing for my project. Before combining all modules i have tested the modules independently, and no errors were reported during testing.
\subsection{Integration Testing}
In this type of testing I have used Big Bang Approach, where all the modules
making up a system are integrated in a single step. In simple words, all the modules of the system are simply put together and tested. However, this technique is practicable only for very small systems. The main problem with this approach is that once an error is found during the integration testing, it is very difficult to localize the error as the error may potentially belong to any of the modules being integrated. Therefore, debugging errors reported during big bang integration testing are very expensive to fix. During this testing no errors were reported and the system worked fine.
\subsection{System Testing}
System tests are designed to validate a fully developed system to assure that it meets its requirements. I have performed this testing to the software and all the requirements that were kept in mind before the development of this system are fulfilled. The system as whole worked as expected and also no errors or problems were reported.
\newpage
\section{Project Screenshots}
\image{0.31}{images/home.png}{Home Page}
\image{0.37}{images/classcalender}{Class Calender}
\image{0.5}{images/signupstudent.png}{Student Signup}
\image{0.5}{images/signupteacher.png}{Teacher Signup}
\image{0.43}{images/studentdashboard.png}{Student Dashboard}
\image{0.4}{images/teacherdashboard.png}{Teacher Dashboard}
\image{0.4}{images/addassignment.png}{Add Assignment}
\image{0.4}{images/submitassignment.png}{Submit Assignment}
\image{0.5}{images/sendmessage.png}{Send Message}
\image{0.5}{images/viewprogress.png}{View Progress}
\image{0.37}{images/admindashboard.png}{Admin Dashboard}
\image{0.45}{images/addstudent.png}{Add New Student}
\newpage
